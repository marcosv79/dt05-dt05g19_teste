\documentclass[a4wide]{report}
\usepackage[utf8]{inputenc}
\usepackage[T1]{fontenc}
\usepackage[portuguese]{babel}
\usepackage{caption}
\usepackage{subcaption}
\usepackage{url}
\usepackage{graphicx}
\usepackage[pdf]{graphviz}
\usepackage{tikz, pgfplots}
\usetikzlibrary{positioning}
\usepackage{xcolor}
\usepackage{listings}
\lstset{basicstyle=\small,
        captionpos=b,
        float=tp,
        xleftmargin=1.5em,
        xrightmargin=1.5em,
        frameround=ttff,
        frame=single,
       floatplacement=tbp}
\def\lstlistingname{Listagem}

\usepackage{amsmath}

\title{Laboratórios de Informática\\ 2022/2023}
\author{Marcos Vasconcelos\\Diogo Oliveira\\Tomás Ferreira}
\date{ 3 de janeiro de 2023 } 
\begin{figure}
\centering
    \includegraphics[width=0.60\textwidth]{transferir.png}
\end{figure}

\begin{document}
\maketitle

\begin{abstract}
    Neste documento irá ser desenvolvido o Trabalho Prático de Matemática Discreta considerando a sua conversão para o LaTeX.\\O trabalho foi desenvolvido pelos alunos Marcos Vasconcelos, Diogo Oliveira e Tomás Ferreira no âmbito da disciplina de Laboratórios de Informática.
\end{abstract}

\tableofcontents
\listoffigures
\listoftables

\chapter{Introdução}
\label{cap:introducao}
Este trabalho representa a conversão de um relatório da disciplina de Matemática Discreta para LaTeX.\\Este relatório foi desenvolvido através do Overleaf, que é um editor online de texto em LaTeX onde é possível editar diversos modelos de documentos, como relatórios e currículos. Também permite a partilha dos projetos com outros utilizadores, sem ser necessário instalar algo.

\section{Objetivos}
Este trabalho prático tem como objetivo elaborar um relatório tendo por base a ferramenta de edição
LaTeX.\\Considerando o documento exemplo desenvolvido ao longos das aulas dedicadas a esta
temática, pretende-se que o documento inclua:
\begin{itemize}
    \item Uma correta estruturação do documento com capítulos, secções e subsecções;
    \item Estruturação de partes do texto com itens, enumerado e descrições;
    \item Uso das diferentes formatações disponíveis, como por exemplo:
    \begin{itemize}
        \item Enfâse;
        \item Itálico;
        \item Negrito.
    \end{itemize}
    \item Inclusão de vários tipos de objetos no documento (figuras, tabelas, listagens de código),
considerando os parâmetros disponíveis para a sua formatação;
    \item Uso de índices dos vários itens incluídos no documento: índice geral, de figuras e de tabelas.
\end{itemize}

\section{Estrutura do documento}
Este documento está dividido em  capítulos:\\O Capítulo \ref{cap:introducao} tem uma breve introdução e objetivos do trabalho.\\No Capítulo \ref{cap:recInfo} tem uma primeira abordagem ao trabalho, onde houve uma seleção e recolha de informação para o desenvolvimento do mesmo.\\No Capítulo \ref{cap:questoes} é onde tem as questões pedidas pelo docente e as respetivas respostas.\\E finalmente no Capítulo \ref{cap:conclusao} temos a conclusão onde falamos dos principais contributos e dificuldades na realização deste trabalho.

\chapter{Recolha de informação}
\label{cap:recInfo}
\section{Escolha de cidades}
Para a resolução deste problema foram selecionadas 9 cidades para detalhar as rotas/destinos aéreos e posteriormente responder a uma série de questões.\\
As cidades escolhidas foram:
\begin{itemize}
    \item Porto (OPO); 
    \item Lisboa (LIS); 
    \item Faro (FAO);
    \item Agadir (AGA); 
    \item Madrid (MAD); 
    \item Dublin (DUB); 
    \item Manchester (MAN); 
    \item Paris (BVA); 
    \item Roma (CIA); 
\end{itemize}

\begin{figure*}[t!]
    \centering
    \begin{subfigure}[t]{0.49\textwidth}
        \centering
        \includegraphics[width=\textwidth]{opo.png}
            \end{subfigure}
            \begin{subfigure}[t]{0.49\textwidth}
        \centering
        \includegraphics[width=\textwidth]{lis.png}
            \end{subfigure}
        \caption{Rotas aéreas: OPO e LIS}
\end{figure*}

\begin{figure*}[t!]
    \centering
    \begin{subfigure}[t]{0.49\textwidth}
        \centering
        \includegraphics[width=\textwidth]{fao.png}
            \end{subfigure}
            \begin{subfigure}[t]{0.49\textwidth}
        \centering
        \includegraphics[width=\textwidth]{aga.png}
            \end{subfigure}
            \caption{Rotas aéreas: FAO e AGA}
\end{figure*}

\begin{figure*}[t!]
    \centering
    \begin{subfigure}[t]{0.49\textwidth}
        \centering
        \includegraphics[width=\textwidth]{mad.png}
            \end{subfigure}
            \begin{subfigure}[t]{0.49\textwidth}
        \centering
        \includegraphics[width=\textwidth]{dub}
            \end{subfigure}
            \caption{Rotas aéreas: MAD e DUB}
\end{figure*}

\begin{figure*}[t!]
    \centering
    \begin{subfigure}[t]{0.49\textwidth}
        \centering
        \includegraphics[width=\textwidth]{man.png}
            \end{subfigure}
            \begin{subfigure}[t]{0.49\textwidth}
        \centering
        \includegraphics[width=\textwidth]{bva.png}
            \end{subfigure}
            \caption{Rotas aéreas: MAN e BVA}
\end{figure*}

\begin{figure*}[h]
    \centering
    \begin{subfigure}[t]{0.49\textwidth}
        \centering
        \includegraphics[width=\textwidth]{cia.png}
            \end{subfigure}
            \caption{Rotas aéreas: CIA}
\end{figure*}

\chapter{Questões}
\label{cap:questoes}
\begin{description} 
\item[]\textbf{1. Represente a situação escolhida por meio de um grafo.}
\end{description}

\begin{tikzpicture}
\node[](a){};
\node[circle,draw](b)[right=of a]{DUB};
\node[circle,draw](c)[right=of b]{MAN};
\node[](d)[right=of c]{};
\node[](e)[right=of d]{};
\node[circle,draw](f)[below=of a]{LIS};
\node[](g)[below=of b]{};
\node[](h)[below=of c]{};
\node[circle,draw](i)[below=of d]{BVA};
\node[](j)[below=of e]{};
\node[](k)[below=of f]{};
\node[circle,draw](l)[below=of g]{OPO};
\node[](m)[below=of h]{};
\node[](n)[below=of i]{};
\node[circle,draw](o)[below=of j]{CIA};
\node[circle,draw](p)[below=of k]{AGA};
\node[](q)[below=of l]{};
\node[](r)[below=of m]{};
\node[circle,draw](s)[below=of n]{MAD};
\node[](t)[below=of o]{};
\node[](u)[below=of o]{};
\node[](v)[below=of p]{};
\node[](w)[below=of q]{};
\node[circle,draw](x)[below=of r]{FAO};
\node[](y)[below=of s]{};
\draw[-](b)--(c);
\draw[-](b)--(i);
\draw[-](b)--(s);
\draw[-](b)--(x);
\draw[-](b)--(l);
\draw[-](b)--(p);
\draw[-](b)--(f);
\draw[-](c)--(i);
\draw[-](c)--(s);
\draw[-](c)--(x);
\draw[-](c)--(l);
\draw[-](c)--(f);
\draw[-](i)--(x);
\draw[-](i)--(l);
\draw[-](o)--(f);
\draw[-](o)--(l);
\draw[-](o)--(p);
\draw[-](s)--(f);
\draw[-](s)--(l);
\draw[-](s)--(p);
\draw[-](s)--(x);
\draw[-](x)--(l);
\draw[-](p)--(l);
\draw[-](p)--(f);
\draw[-](f)--(l);
\end{tikzpicture}

\begin{description} 
\item[]\textbf{2. Indique, justificando, se o grafo é conexo.\\
No contexto do problema, interprete a sua resposta.
}\\
O grafo é conexo porque há uma cadeia entre quaisquer 2 vértices.\\
No contexto do problema, interpreta-se que é possível chegar a qualquer cidade seja qual for a cidade de partida.\\\\\\\\\\
\end{description}

\begin{description} 
\item[]\textbf{3. Indique, justificando, se o grafo é eureliano.\\
No contexto do problema, interprete a sua resposta.
}\\
O grafo não é eureliano porque nem todos os vértices têm grau par. Não é possível desenhar o grafo sem repetir linhas e sem levantar o lápis.\\
No contexto do problema, interpreta-se que não é possível fazer todos os voos sem repetir mais do que uma vez pelo menos um voo.\\
\end{description}

\begin{description} 
\item[]\textbf{4. Indique, justificando, se o grafo é hamiltoniano.\\
No contexto do problema, interprete a sua resposta.
}\\
O grafo é hamiltoniano porque é possível passar por todos os vértices sem ter de os repetir.\\
No contexto do problema, é possível ir a todas as cidades sem repetir cada uma delas.\\
\end{description}

\begin{description} 
\item[]\textbf{5. Apresente a matriz de adjacência do grafo considerado na questão 1.\\
}
\end{description}
\begin{table}[h]
    \centering
    \begin{tabular}{|c|c|c|c|c|c|c|c|c|c|}
        \hline
        \ & OPO & LIS & FAO & AGA & MAD & DUB & MAN & BVA & CIA \\
        \hline
        OPO & 0 & 1 & 1 & 1 & 1 & 1 & 1 & 1 & 1 \\
        \hline
        LIS & 1 & 0 & 0 & 1 & 1 & 1 & 1 & 0 & 1 \\
        \hline
        FAO & 1 &0 &0& 0& 1& 1& 1 &1& 0 \\
        \hline
        AGA & 1& 1 &0 &0 &1& 1& 0& 0& 1 \\
        \hline
        MAD &1& 1& 1& 0& 0 &1& 1& 0& 0 \\
        \hline
        DUB& 1 &1& 1 &1 &1& 0& 1& 1 &0 \\
        \hline
        MAN& 1& 1& 1 &0& 1& 1 &0 &1 &0 \\
        \hline
        BVA& 1& 0& 1& 0& 0& 1 &1& 0& 0 \\
        \hline
        CIA& 1 &1& 0& 1& 0& 0& 0& 0& 0 \\
        \hline
    \end{tabular}
    \caption{Matriz de adjacência}
\end{table}
\begin{description} 
\item[]\textbf{\\\\\\\\\\\\\\\\\\\\\\\\\\6. Indique o fecho transitivo direto de um vértice à sua escolha.\\
No contexto do problema, apresente uma interpretação para o conjunto obtido.\\
}
No contexto do problema, é possível interpretar que com apenas 1 voo de partida em Lisboa (LIS) podemos chegar a qualquer outra cidade (exceto Faro (FAO) e Paris (BVA)).
\end{description}
\begin{table}[h]
    \centering
    \begin{tabular}{|c|}
        \hline
        LIS \\
        \hline
        1 \\
        \hline
        0 \\
        \hline
        2 \\
        \hline
        1 \\
        \hline
        1 \\
        \hline
        1 \\
        \hline
        1 \\
        \hline
        1 \\
        \hline
        2 \\
        \hline
        1 \\
        \hline
    \end{tabular}
        \caption{Fecho transitivo direto do vértice LIS}
\end{table}

\begin{description} 
\item[]\textbf{\\\ 7. Indique o fecho transitivo inverso de um vértice à sua escolha.\\
No contexto do problema, apresente uma interpretação para o conjunto obtido.\\
}
No contexto do problema, é possível interpretar que em quase todas as cidades (exceto Porto (OPO), Lisboa (LIS) e Agadir (AGA)) é necessário fazer 2 voos para chegar ao aeroporto de Roma (CIA).
\end{description}
\begin{table}[h]
    \centering
    \begin{tabular}{|c|c|c|c|c|c|c|c|c|c|c|}
        \hline
        CIA & 1 & 1 & 2 & 1 & 2 & 2 & 2 & 2 & 2 & 2 \\
        \hline
    \end{tabular}
            \caption{Fecho transitivo inverso do vértice CIA}
\end{table}

\chapter{Conclusão}
Neste trabalho abordámos o assunto de fazer a edição de um documento através do editor de texto LaTeX e concluímos que foi uma excelente experiência e cumprimos com todos os objetivos que nos tínhamos proposto.\\
Foi um trabalho muito importante para o nosso conhecimento e aprofundamento deste tema, uma vez que nos permitiu aperfeiçoar e desenvolver competências de investigação, seleção e organização. 
\label{cap:conclusao}
\section{Principais contributos}
Houveram várias contributos na realização deste trabalho. Conhecemos uma nova ferramenta de edição de texto que nos era desconhecida que nos trouxe várias vantagens:
\begin{enumerate}
    \item Conhecimento de um novo editor de texto;
    \item O LaTeX normalmente não tem bugs, ao contrário do editor de texto Word;
    \item A linguagem LaTeX é indicada para escrever artigos e trabalhos académicos de todo o tipo;
    \item As versões dos arquivos não sofrem com incompatibilidade.
\end{enumerate}
\section{Dificuldades}
Também houveram algumas dificuldades, tais como:
\begin{itemize}
    \item Aprendizagem da linguagem no ínicio do relatório;
    \item Paciência e dedicação para dominar a linguagem utilizada no LaTeX;
    \item Correção de erros;
    \item Criação de grafos.
\end{itemize}

\bibliographystyle{alpha}
\bibliography{bibliografia}
 JabRef - https://www.jabref.org/\\Zotero - https://www.zotero.org/
\end{document}